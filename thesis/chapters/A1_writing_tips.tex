\chapter{Tips for writing a thesis}

\section{Write down everything, you know}
It is recommended, that you use this template to write down everything, that is relevant for the thesis, as soon as that piece of information is available to you.
Also, write down the topics as work items, that require further research or development.

This method has a couple of advantages compared to the habit of delaying the writing to the very end:
\begin{itemize}
	\item You automatically keep track of the work, you have done, so you are less likely to forget to include anything at the end.
	\item You automatically keep track of the work, that still needs to be done, which reduces the risk of constantly changing objectives.
	\item The structure of the thesis reminds you, in which order, the work items have to be addressed.
	      So, for example, you are less likely to start programming without doing a requirements analysis, first.
	\item You get an overview about the extent of the thesis much earlier, which improves the estimation, how much time is necessary to finalize the writing.
	\item You always have a documentation, that you can discuss with the supervisors.
\end{itemize}

While it is recommended, that you include content as early as possible, it is also recommended, to wait with writing out the thesis as long as possible.
Keeping a draft of the thesis in bullet points makes it much easier to add new information or to rearrange the existing content.
Also, as long as the bullet points are well written, they require a lot less work from the supervisor to read, which lowers the bar to discuss the current progress with her/him.
So take care, that the bullet points are comprehensible for other people.

Make sure, that you have really included everything, that you currently know.
For example, a rhetorical question is an indication for missing information, because you wrote a question, where you should have written the answer, that you already know.

\section{Do one thing, but do it well}
Make sure, that you do not jump between topics.
\begin{itemize}
	\item A paragraph shall only make one statement.
	\item A (sub)subsection shall only comprise one coherent sequence of such statements.
	\item A section shall only contain topics, that are closely related.
	\item A chapter shall only contain sections, that describe the same feature of the thesis.
\end{itemize}

Resist the temptation to include additional information, that is not required in the current context.
Even if it is only one adjective, the reader will wonder, why it is important, which will confuse and distract her/him.

\section{If in doubt, show an image}
Charts and diagrams are a great way to convey an intuitive understanding of the topic at hand.
Don't forget to include a lot of them.

\section{Don't do manual postprocessing}
The human intuition is easily fooled, which can lead to wrong conclusions.
So if we want to interpret data, we must make sure, that we present it in a way, that can be interpreted directly without any additional processing steps.
For example:

\begin{itemize}
	\item Do not compare two values by looking at the numbers.
	      Plot them and look at the chart.
	      And make sure to use the right scale; e.g. most values about the human perception must be evaluated on a logarithmic scale.
	\item Do not compare two lines in a chart for similarity.
	      Compute the difference between the two and plot that.
		  Maybe compute the absolute values before plotting.
	      And if possible, include a horizontal line for some threshold, so that the reader does not forget to consider the y-values.
\end{itemize}

The problem with looking at numbers is, that we prone to confuse the visual difference of two numbers with the actual difference of the value.
For example 168.3 sems to be much closer to 122.7 than 77.3, because both 168.3 and 122.7 have the same number of digits and the decimals have to be rounded to reduce the distance between the two.

Even charts are not fool-proof.
When comparing two plot lines, we tend to compare the direct distance between the two lines.
When the lines have a very steep slope, this distance is substantially lower than the vertical distance between two y-values, that correspond to the same x-value.

Filtering unrelated data is also considered postprocessing, so the \textit{plot one thing, but plot it well} rule applies here, too.

