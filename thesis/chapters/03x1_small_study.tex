\section{A small study and an analysis}
The most common type of thesis in the digital health sector is probably one, in which the student conducts a small study and uses the results to build a mathematical model.
The purpose of the model is usually either to predict a health-related event or to classify a medical condition.
In order to be able to conduct the study, the thesis often includes the development of an app or another piece of software.

\subsection{Method}
This chapter shall describe the analysis method, that is applied to the results of the study.
Justify, why you have chosen this method and which other methods you have considered (maybe, this justification is better put after the description of the method).

Give a comprehensible description of the method.
If it is necessary to derive this method from other approaches, make sure, that you do not repeat content from the \textit{State of the art} chapter.

\subsection{Study}
This chapter shall describe the study, that has been conducted.
It should start with the goal of the study.
Do not repeat too much of the introduction though, since restating the goal here is mainly to facilitate the justification of the study design.

The next thing to describe is the study design.
List, what data is being recorded over which period of time.
Justify these decisions by referring to the analysis method, that is described in the previous chapter.
Make a statement about how many participants are needed to get significant results.
If the number of participants, that can be realistically expected, is lower than the required one, explain, how you address that issue.
Describe, if all participants are treated equally by the study or if you for example have an intervention group and a control group.
Make an assumption on how the recording of the data influences the study participants and explain, what measures you have taken to get realistic results.

And finally, you must report, how your particular study went.
This is mostly about giving the numbers in order to give the reader an idea, how big the study was and if it turned out as planned.
The analysis of the data is being done in later chapters.
Describe, when the study was conducted, if it was conducted at a special location and how many people participated.
It might also be informative to state a few simple values, that quantify the amount of data, that has been recorded.
Legal issues, for example, whether the study was approved by an ethics commission, must also be mentioned here.

\subsection{Software development}
This chapter shall describe the software, that has been used for the study, and its development.
It roughly follows the structure, that is recommended for a system development thesis, but the content is condensed into a single chapter.

The chapter should start with describing the purpose of the software and deriving the requirements from that.
The documentation of the requirements analysis shall comprise references to the study design, which justify the listed requirements.

After that, the software architecture shall be explained.
Justify your design choices by stating, which requirements you intent to meet with them.
Do not explain your particular implementation in too much detail.
A rule of thumb is, that this chapter shall enable another software developer to reimplement your software in a different programming language.
So describe the subsystem composition and maybe discuss a few particularly interesting design choices on a class- or interface-level.
Showing source code is usually discouraged.
Be careful, not to come up with new requirements in this section.
For example, if you describe an implementation detail, that improves the software's performance, make sure, that performance is also listed in the requirements.

Then, you shall describe your particular implementation.
Often, this can be very short.
Simply mention the target platform, the programming language and the most important frameworks and justify these selections with the requirements.
Some features, that always have to be discussed in greater detail, are those with possible legal consequences.
So, if the study required special care for the participants' privacy, include a documentation of the software's concept of how the access to the data is restricted.
This description shall include major technical details, such as the applied encryption algorithms, so the reader can verify, if this concept corresponds to the current state of the art.
The same applies to features, which improve data security and fault tolerance.

And finally, the running software shall be presented.
Give a short summary of the workflow of the most important usage scenarios and show some screen shots.

\subsection{Results}
In this chapter, you shall present the results of the study.
Provide the reader with an intuition of how to interpret the data, so she/he is better equipped to follow the subsequent \textit{Analysis} chapter.

\subsection{Analysis}
Describe, how you have applied the previously described method to the results of your study.
Then present the results of this analysis.
And finally assess, how plausible these results are.

Make sure, that you are not only presenting numbers, but that you explain their meaning, too.
The reader must be able to interpret the results and she/he must get an intuition, what impact these results have on the field of research.

