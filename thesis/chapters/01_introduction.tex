\chapter{Introduction}

Introduce the topic of your thesis.

Usually, this chapter is only finalized at the very end of writing a thesis, so the summary of the own work and the references to other works can be updated.
However, it is advised, that the problem, motivation and objective sections are drafted in bullet points very early in the thesis, in order to have a fixed objective for the thesis.

For theses, that do not have a lot of related research, which must be referenced, it might be better to leave out the \textit{State of the art/Related research} chapter and include these references in the \textit{Motivation} and \textit{Objective} sections in this chapter.

\section{Problem}

Describe the problem, that you address in your thesis.
For this, it is probably necessary to describe the context, in which the research has taken place.

\section{Motivation}

Justify scientifically why solving the problem of your thesis is necessary.
Have in mind, that the motivation for the thesis can come from both, deficiencies in the current state of the art and benefits from having a solution to the problem of the thesis.
Be careful, not to be reproachful, when describing a lack in the current state of the art.

Since such a justification is also likely to require a description of the broader context, it might not make sense to separate this section from the problem section.
Nevertheless, even if you join the sections, try to describe the problem, before assessing its severity.

\section{Objectives}

State the research goals of your thesis.
Also, list and illustrate the methods, that you are using for your research.

\section{Outline}

Give a short overview, where to find the most important topics in your thesis.
Make sure, that this is just not an enumeration of the table of contents.

