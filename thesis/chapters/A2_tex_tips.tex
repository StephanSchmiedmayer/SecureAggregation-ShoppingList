\chapter{Tips for writing a thesis in TeX}

\section{using this template}
This template tries to achieve a separation of the template itself and the parts that are specific to the thesis.
Ideally, the template itself does not have to be edited (that means the \texttt{thesis.tex} file and the files in the \texttt{template}-folder).

The content of the thesis shall be added to the following files and folders:
\begin{itemize}
	\item the \texttt{.tex}-files in the \texttt{chapters}-folder shall contain the description of your work.
	\item the \texttt{.tex}-files in the \texttt{resources}-folder contain templates and examples, into which metadata, settings and organisational information about the thesis can be entered.
	\item the \texttt{thesis.bib}-file shall contain a list of the literature, that you cite in your thesis.
\end{itemize}

\section{General tips}
Track your work on this thesis with a version control system such as git.

In your TeX source code, use one line per sentence.
This facilitates spotting excessively long sentences.
Also, it makes the tracking of changes by the version control system more useful.
If you add line breaks after a fixed number of columns instead, a change affects all subsequent lines of the paragraph, even though the actual contend has not been changed.

It is recommended to create a folder, in which all images, that are included in this document are stored.
See the \texttt{resources/settings.tex}-file, on how to add this folder to the default graphics path, so only the filenames have to be entered, when including an image.

