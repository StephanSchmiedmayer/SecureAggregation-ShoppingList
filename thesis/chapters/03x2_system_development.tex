\section{Development of a system}
In some theses, the focus is on developing a system.
Usually, this is a software system, but it can include the development of a piece of hardware, too.
In such a thesis, there is much more emphasis on the development process, than on the analysis of data.
Therefore, the design process must be documented extensively in the thesis.

Prof. Brügge wrote a book about software engineering, that might be applicable to this thesis \cite{bruegge2009object}.
The recommended chapters in this template are derived from this book.

\subsection{Requirements Analysis}
	This chapter shall document the requirements of the system, that shall be developed.
	These requirements shall later be used to justify the design decisions for the system's implementation

	\subsubsection{Overview}
		Provide a short overview about the purpose, scope, objectives and success criteria of the system that you like to develop.
	\subsubsection{Current System}
		If the system, that is developed in this thesis shall replace an existing system, the existing system shall be described here.
		Make sure, that there are no repetitions from the \textit{State of the art} chapter.
	\subsubsection{Proposed System}
		If you leave out the section \textit{Current system}, you can rename this section to \textit{Requirements}.
		\paragraph{Functional Requirements}
			List and describe all functional requirements of your system.
			Also mention requirements that you were not able to realize.
		\paragraph{Nonfunctional Requirements}
			List and describe all nonfunctional requirements of your system.
			Also mention requirements that you were not able to realize.
			Categorize them using the FURPS+ model described in \cite{bruegge2009object} without the category \textbf{functionality} that was already covered with the functional requirements.
	\subsubsection{System Models}
		This section includes important system models for the requirements analysis.
		If you do not distinguish between visionary and demo scenarios, you can just add a \textit{Scenarios} section.
		\paragraph{Visionary Scenarios}
			Describe one or two visionary scenarios here, i.e. a scenario that would perfectly solve your problem, even if it might not be realizable.
			Use the scenario description template from the ASE-Chair in form of a table.
		\paragraph{Demo Scenarios}
			Describe one or two demo scenario here, i.e. a scenario that you can implement and demonstrate until the end of your thesis.
			Use the scenario description template from the ASE-Chair in form of a table.
		\paragraph{Use Case Model}
			This subsection should contain a UML Use Case Diagram including a description of the roles and their use cases.
			You can use colors to indicate priorities.
			Think about splitting the diagram into multiple ones if you have more than 10 use cases.
			\textbf{Important:} Make sure to describe the most important use cases using the use case table template.
			Also describe the rationale of the use case model, i.e. why you modeled it like you show it in the diagram.
		\paragraph{Analysis Object Model}
			This subsection should contain a UML Class Diagram showing the most important objects, attributes, methods and relations of your application domain including taxonomies using specification inheritance (see \cite{bruegge2009object}).
			Do not insert objects, attributes or methods of the solution domain.
			\textbf{Important:} Make sure to describe the analysis object model thoroughly in the text so that readers are able to understand the diagram.
			Also write about the rationale how and why you modeled the concepts like this.
		\paragraph{Dynamic Model}
			This subsection should contain dynamic UML diagrams.
			These can be UML state diagrams, UML communication diagrams or UML activity diagrams.
			\textbf{Important:} Make sure to describe the diagram and its rationale in the text.
			\textbf{Do not use UML sequence diagrams.}
		\paragraph{User Interface}
			Show mockups of the user interface of the software you develop and their connections/transitions.
			You can also create a storyboard.
			\textbf{Important:} Describe the mockups and their rationale in the text.

\subsection{System Design}
	In this chapter, you describe, how you map the concepts of the application domain to the solution domain.
	Some sections are optional, if they do not apply to your problem.
	\paragraph{Overview}
		Provide a brief overview of the software architecture and references to other chapters (e.g. requirements analysis), references to existing systems, constraints impacting the software architecture.
	\paragraph{Design Goals}
		Derive design goals from your nonfunctional requirements, prioritize them (as they might conflict with each other) and describe the rationale of your prioritization.
		Any trade-offs between design goals (e.g., build vs. buy, memory space vs. response time) and the rationale behind the specific solution should be described in this section
	\paragraph{Subsystem Decomposition}
		Describe the architecture of your system by decomposing it into subsystems and the services provided by each subsystem.
		Use UML class diagrams including packages/components for each subsystem.
	\paragraph{Hardware Software Mapping}
		This section describes, how the subsystems are mapped onto existing hardware and software components.
		The description is accompanied by a UML deployment diagram.
		The existing components are often off-the-shelf components.
		If the components are distributed on different nodes, the network infrastructure and the protocols are also described.
	\paragraph{Persistent Data Management}
		An optional section that describes, which and how data is saved over the lifetime of the system.
		Describe the approach for persisting data here and show a UML class diagram of how the entity objects are mapped to persistent storage.
		Also, justify the selected storage technology and data model.
	\paragraph{Access Control}
		An optional section describing the access control and security issues based on the nonfunctional requirements of the system.
		It also describes the implementation of the access matrix based on capabilities or access control lists, the selection of authentication mechanisms and the use of encryption algorithms.
	\paragraph{Global Software Control}
		An optional section describing the control flow of the system, in particular, whether a monolithic, event-driven control flow or concurrent processes have been selected, how requests are initiated and specific synchronization issues.
	\paragraph{Boundary Conditions}
		An optional section describing the use cases how to start up the separate components of the system, how to shut them down, and what to do if a component or the system fails.

\subsection{Case Study/Evaluation}
	If you did an evaluation/case study, describe it here.
	\paragraph{Design}
		Describe the design/methodology of the evaluation and why you did it like that.
		E.g. what kind of evaluation have you done (e.g. questionnaire, personal interviews, simulation, quantitative analysis of metrics, what kind of participants, what kind of questions, what was the procedure?
	\paragraph{Objectives}
		Derive concrete objectives/hypotheses for this evaluation from the general ones in the introduction.
	\paragraph{Results}
		Summarize the most interesting results of your evaluation (without interpretation).
		Additional results can be put into the appendix.
	\paragraph{Findings}
		Interpret the results and conclude interesting findings.
	\paragraph{Discussion}
		Discuss the findings in more detail and also review possible disadvantages that you found.
	\paragraph{Limitations}
		Describe limitations and threats to validity of your evaluation, e.g. reliability, generalizability, selection bias, researcher bias.

